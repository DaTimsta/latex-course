\documentclass[a4paper,oneside]{memoir}
\usepackage[utf8]{inputenc}
\usepackage[english]{babel}
\usepackage{csquotes}
\usepackage{geometry}
\usepackage{url}
\counterwithout{section}{chapter}
\usepackage{xcolor}
\usepackage{listings}
\setlength\parskip{0.1 cm}
\setlength\parindent{0pt}

\lstset
{
    language=[LaTeX]TeX,
    breaklines=true,
    basicstyle=\ttfamily,
    keywordstyle=\color{blue},
    identifierstyle=\color{black},
    morekeywords={chapter, subsection, subsubsection, chapterstyle}
}


\begin{document}
\section{Compiling your first document}
Create a .tex file in Texmaker and paste the following lines into the empty file:

\begin{quote}
\begin{lstlisting}
\documentclass{memoir}
\begin{document}
Hello world!
\end{document}
\end{lstlisting}
\end{quote}

\noindent Try compiling the document. Do you see a pdf file with the words \enquote{Hello World!}?

\section{Basic text formatting}

\begin{itemize}
\item \textbf{Bold} text is done using \lstinline$\textbf{text}$
\item \emph{Italic} (or emphasized) text is done using \lstinline$\emph{text}$
\end{itemize}

For anything to show up in your document, it needs to be placed between \lstinline$\begin{document}$ and \lstinline$\end{document}$! Play around with these commands and ensure that they work.

\section{Chapters and sections}
To structure your document, use \LaTeX{} commands to create chapters and sections. A chapter is created using \lstinline$\chapter{text}$ and a section with \lstinline$\section{text}$. Chapters are divided into sections and are numbered as such. For more levels of structure, you can use \lstinline$\subsection{text}$ and \lstinline$\subsubsection{text}$. To prevent a chapter from showing up in the table of contents use \lstinline$\chapter*{text}$ and equivalently for sections etc.

Create three chapters, \enquote{Abstract}, \enquote{Introduction} and \enquote{Theory}. Make sure the abstract isn't listed in the table of contents. Make some sections in the introduction in order to understand the concepts. Add the line \lstinline$\tableofcontents$ just after \lstinline$\begin{document}$ to generate a table of contents. 

Fill out the document with some text to see how it looks. Use your own or generate some dummy text using \url{http://www.lipsum.com/}

\section{Bullet points}

\section{Adding some formatting}
Everything before \lstinline$\begin{document}$ is generally used to define some general characteristics about your document. This part of the document is known as the \enquote{preamble}. Try adding the following lines to your document (after \lstinline$\documentclass$ and before \lstinline$\begin{document}$:

\begin{quote}
\begin{lstlisting}
\usepackage{geometry}
\chapterstyle{section}
\usepackage{microtype}
\usepackage[english]{babel}
\end{lstlisting}
\end{quote}

Also change the documentclass to \lstinline$\documentclass[a4paper,oneside]{memoir}$.

\section{Collaborative work}

\section{Introduction toreferences}



\end{document}