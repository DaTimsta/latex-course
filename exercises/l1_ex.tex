\documentclass[a4paper,oneside]{memoir}
\usepackage[utf8]{inputenc}
\usepackage[english]{babel}
\usepackage{csquotes}
\usepackage{geometry}
\usepackage{url}
\counterwithout{section}{chapter}
\usepackage{xcolor}
\usepackage{listings}
\setlength\parskip{0.1 cm}
\setlength\parindent{0pt}

\lstset
{
    language=[LaTeX]TeX,
    breaklines=true,
    basicstyle=\ttfamily,
    keywordstyle=\color{blue},
    identifierstyle=\color{black},
    morekeywords={chapter, subsection, subsubsection, chapterstyle, maketitle, setlength,
    			  tableofcontents, subfile}
}


\begin{document}

\section{Compiling your first document}
Create a  new file in Texmaker and and save it as \texttt{main.tex}. Type the following lines into the empty file:

\begin{quote}
\begin{lstlisting}
\documentclass{memoir}
\begin{document}
Hello world!
\end{document}
\end{lstlisting}
\end{quote}

\noindent Try building the document by selecting \enquote{Tools} and then \enquote{Quick Build} in the menu bar. Do you see a pdf file with the words \enquote{Hello World!}?

\section{Basic text formatting}

\begin{itemize}
\item Normal text is done by simply writing it as-is
\item \textbf{Bold} text is done using \lstinline$\textbf{text}$
\item \emph{Italic} (or emphasized) text is done using \lstinline$\emph{text}$
\end{itemize}

For anything to show up in your document, it needs to be placed between \lstinline$\begin{document}$ and \lstinline$\end{document}$! Play around with these commands, type in some text and build your document again.

\section{Chapters and sections}
To structure your document, we use \LaTeX{} commands to create chapters and sections. A chapter is created using \lstinline$\chapter{text}$ and a section with \lstinline$\section{text}$. Chapters start on a new page and are divided into sections and are numbered as such. For more levels of structure, you can use \lstinline$\subsection{text}$ and \lstinline$\subsubsection{text}$. To prevent a chapter from showing up in the table of contents use \lstinline$\chapter*{text}$ and equivalently for sections etc.

Create three chapters, \enquote{Abstract}, \enquote{Introduction} and \enquote{Theory} in your \texttt{main.tex}. Make sure the abstract isn't listed in the table of contents. Make some sections in the introduction in order to understand the concepts. Add the line \lstinline$\tableofcontents*$\footnote{The asterisks is to ensure the table of contents isn't listed in the table of contents!} just after \lstinline$\begin{document}$ to generate a table of contents.

Fill out the chapters and sections with some text. Use your own or generate some dummy text using \url{http://www.lipsum.com/}. Your \texttt{main.tex} should now look something like this:

\begin{quote}
\begin{lstlisting}
\documentclass{memoir}
\begin{document}
\tableofcontents*
\chapter*{Abstract}
content of abstract
\chapter{Introduction}
content of introduction
\section{Project structure}
content of the project structure
\chapter{Theory}
content of theory
\end{document}
\end{lstlisting}
\end{quote}

\section{Bullet points}
To make a list of bullet points, add the following to your document in the desired location:

\begin{quote}
\begin{lstlisting}
\begin{itemize}
\item item 1
\item item 2
\item item 3
\end{itemize}
\end{lstlisting}
\end{quote}

To have a numbered list, replace \texttt{itemize} with \texttt{enumerate}. Try adding a list of bullet points somewhere in your introduction.

\section{Adding some formatting}
Everything before \lstinline$\begin{document}$ is generally used to define some general characteristics about your document. This part of the document is known as the \emph{preamble}. Try adding the following lines to your document (after \lstinline$\documentclass$ and before \lstinline$\begin{document}$):

\begin{quote}
\begin{lstlisting}
\usepackage{geometry}
\usepackage{microtype}
\usepackage[english]{babel}
\chapterstyle{section}
\end{lstlisting}
\end{quote}

Also change the documentclass to \lstinline$\documentclass[a4paper,oneside]{memoir}$. 

The \lstinline$\usepackage$ command loads packages for use in our document. The \texttt{geometry} package helps us by setting some good defaults for the geometry of the page -- margins etc. The \texttt{microtype} package improves the rendering of fonts. The \texttt{babel} package offers language support and makes sure that automatically generated words, date/time, hyphenation patterns etc. reflect the language of the document. When writing a danish report, simply change the parameter \texttt{english} to \texttt{danish}.

The \lstinline$\chapterstyle$\footnote{this is a feature of the memoir class and will not work with standard \LaTeX{} document classes} command, unsurprisingly, sets a style for the chapters in our document. Using the style \texttt{section} simply makes the chapter headline look like the section headlines (albeit a bit bigger). Try changing this style to \texttt{verville} for a different style and see how it affects your document.

The optional parameters \texttt{a4paper} and \texttt{oneside} in the \lstinline$\documentclass$ tells \LaTeX to generate the pdf on a4 paper meant to be printed on one-sided paper. If printing on both sides, you can change \texttt{oneside} to \texttt{twoside}. A two-sided document also makes sure that chapters always start on the right!

\section{Paragraphs}
Paragraphs in \LaTeX{} are separated by \emph{two} line returns. By default, the first paragraph in a chapter, section etc. is not indented, but every following paragraph is. The behaviour of this indentation can be changed in the preamble using the following two commands:

\begin{quote}
\begin{lstlisting}
\setlength\parskip{0.1 cm}
\setlength\parindent{0pt}
\end{lstlisting}
\end{quote}

Changing the length of \lstinline$\parskip$ changes the distance between paragraphs. Lengths in \LaTeX{} accept a variety of units and in this case we simply use centimetres. The length of \lstinline$\parindent$ changes the indentation of the paragraphs. In this case I simply removed it.

\section{Adding a title}
Add the following lines to the preamble:

\begin{quote}
\begin{lstlisting}
\title{Your title}
\author{Your name}
\end{lstlisting}
\end{quote}

And add the command \lstinline$\maketitle$ just after \lstinline$\begin{document}$. If you want the title to be on its own page, add \lstinline$\clearpage$ after \lstinline$\maketile$

\section{Collaborative work}
In order to comfortably collaborate on a report, it can be advantageous to split the document up in a number of files, so people can work on different things without interrupting each other. A great way to do this is by using the \texttt{subfile} package. First load the package by adding \lstinline$\usepackage{subfiles}$ to the preamble. We now want to load the three chapters from earlier into three separate files. Create \texttt{abstract.tex}, \texttt{intro.tex}, \texttt{theory.tex}. These \emph{subfiles} are structured in the following way:

\begin{quote}
\begin{lstlisting}
\documentclass[main.tex]{subfiles}
\begin{document}
Content here
\end{document}
\end{lstlisting}
\end{quote}

Copy over the content to the relevant files. In \texttt{main.tex}, our original file, the subfiles are then loaded using the \lstinline$\subfile$ command. The resulting \texttt{main.tex} should now look like this:

\begin{quote}
\begin{lstlisting}
\documentclass[a4paper, oneside]{memoir}
\usepackage{geometry}
\usepackage{microtype}
\usepackage{subfiles}
\usepackage[english]{babel}
\setlength\parskip{0.1 cm}
\setlength\parindent{0pt}
\title{Your title}
\author{Your name}
\chapterstyle{section}

\begin{document}
\maketitle
\tableofcontents*
\subfile{abstract.tex}
\subfile{intro.tex}
\subfile{theory.tex}
\end{document}
\end{lstlisting}
\end{quote}

If you build \texttt{main.tex} you will get a \texttt{.pdf} of the entire document. If you build any of the subfiles, you will get a \texttt{.pdf} with the contents of the given chapter. Try it out!

\section{Introduction to references}
We will go into more detail with references later on, but the basics are actually quite simple! Lets say you want to reference a section called \enquote{Thermodynamics} in the \enquote{Theory} chapter. Label the section by writing

\begin{quote}
\begin{lstlisting}
\section{Thermodynamics}\label{sec:thermo}
\end{lstlisting}
\end{quote}

The section \enquote{Thermodynamics} section now has the label \texttt{sec:thermo}. When you want to reference this section, you simply use the command \lstinline$\ref{sec:thermo}$ to get the numerical value (e.g. 1.3) of that section. Usually you would write something like:

\begin{quote}
\lstinline$... as shown in section \ref{sec:thermo} ...$
\end{quote}

In your \texttt{.tex} file. Try adding some labels and references to your document. You may have to build the document twice in order for the references to work correctly. The \LaTeX{} interpreter will usually warn you if this is the case.

\end{document}